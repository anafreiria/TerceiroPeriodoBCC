\section{Introdução}

O Problema do Caixeiro Viajante (PCV), conhecido internacionalmente como \textit{Traveling Salesman Problem} (TSP), é um dos problemas mais estudados na área de Ciência da Computação e pesquisa operacional. Classificado como NP-difícil, o PCV consiste em encontrar o caminho mais curto possível que permita a um caixeiro viajante visitar um conjunto de cidades, passando por cada uma exatamente uma vez, e retornando à cidade de origem.

Matematicamente, o problema pode ser definido sobre um grafo completo $G = (N, E)$, onde:
\begin{itemize}
    \item $N = \{c_1, c_2, \ldots, c_n\}$ representa o conjunto de cidades
    \item $E$ é o conjunto de arestas conectando todos os pares de cidades
    \item Cada aresta $(c_i, c_j)$ possui um peso $p(c_i, c_j)$ que representa a distância euclidiana entre as cidades
\end{itemize}

O objetivo é encontrar um ciclo hamiltoniano (um caminho que visite cada cidade exatamente uma vez e retorne ao ponto de partida) com o menor custo total, ou seja, com a menor soma das distâncias percorridas.

Este trabalho tem como objetivo implementar e comparar três diferentes abordagens algorítmicas para resolver o PCV:
\begin{itemize}
    \item \textbf{Força Bruta}: Abordagem exaustiva que avalia todas as possíveis rotas
    \item \textbf{Algoritmo Guloso}: Heurística que toma decisões localmente ótimas em cada etapa
    \item \textbf{Programação Dinâmica}: Método de Held-Karp que armazena soluções de subproblemas para evitar recálculos
\end{itemize}

A relevância do PCV vai além do interesse teórico, com aplicações práticas em diversas áreas como:
\begin{itemize}
    \item Logística e planejamento de rotas
    \item Projeto de circuitos integrados
    \item Sequenciamento de DNA
    \item Planejamento de movimentos robóticos
\end{itemize}

Neste documento, apresentaremos detalhes sobre cada algoritmo implementado, sua complexidade computacional, e os resultados obtidos em diferentes instâncias do problema retiradas da biblioteca TSPLIB. A comparação entre as abordagens será feita tanto em termos de qualidade da solução (distância total da rota) quanto em termos de eficiência (tempo de execução).